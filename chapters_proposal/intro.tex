\section{Introduction}\label{sec:intro}
Despite the urgency of the topic, our knowledge on causes and mechanisms as well as our ability to reliably predict tree mortality on individual to regional scale is still lacking \citep{mcdowell2013, hartmann2015}. Forest ecosystems worldwide have been observed to show accelerating rates of tree mortality and forest die-off linked to drought stress and increased temperatures \citep{allen2010, chaparro2017, vanmantgen2009}. As climate change continues, this process is likely to intensify \citep{allen2015} putting entire ecosystems at risk to undergo regime shifts. Forests globally are increasingly at the risk of transitioning into savannahs or other alternative states \citep{hirota2011}. This threat concretizes in changes in frequency, intensity, duration and timing of fires and droughts, but also by the introduction of species, outbreaks of insects and pathogens, hurricanes, windstorms, ice storms, or landslides \citep{dale2001}. The phenomenon of tree mortality may occur at the level of the organism, but the processes leading to forest mortality are observable and describable across spatial, organizational and temporal scales and therefore demands for interdisciplinary approaches \citep{hartmann2015}.\\
According to \cite{cailleret2017} tree mortality is preceeded by reduced growth rates in about 84\% of the cases, and thus adding changes over time to potential indications of tree mortality. These changes over time can also be characterized by remotely sensed \gls{resilience} \citep{verbesselt2016}. Remotely sensed \gls{resilience} is based on the concept of Early Warning Signals \citep[EWS,][]{carpenter2006, carpenter2008, dakos2008} that announce tipping points based on slowing down of the system.\\
On the one hand early warning signals could provide the potential to warn in time when an ecosystem reaches a critical state \citep{dakos2014}. On the other hand early warning signals can also be seen as indicators of resilience of a system \citep{dakos2012}. Although EWS might not outperform direct (field) measurements when looking at resilience \citep{vbelzen2017}. But for the above described problem to predict tree mortality, they might support the predictability since the target variable mortality is directly linked to resilience. Remotely sensed resilience could then contribute to large scale predictions and offer the possibility to look back in time \citep{verbesselt2016}. Since methods to warn in time but also mortality prediction approaches are not available at this point, this research has great potential for forest management as based on this certain actions could be taken in advance.\\






%\cite{verbesselt2016} conducted a research on remotely sensed resilience of tropical forests and was able to show that resilience decreases with higher temperatures and lower precipitation. The study showed the vulnerability of forest ecosystems to changing environmental conditions that might cause forest ecosystems to massively transition into alternative stable states. 



