\section{Theory on Tree Mortality and Early Warnings}\label{sec:theory}
Both the processes causing tree mortality and the exact definition of it remain a subject of research \citep{hartmann2015}. From a theoretical point of view tree death occurs at the point of minimum vitality \citep{dobbertin2005, grivcar2012}. From an anatomical and developmental perspective this point is less clear \citep{schenk2008}: one or more organs and tissues of the tree can for instance fatally desiccate when at the same time, other tissues like apical, cambial and/or root meristems might still be intact and keep a plant alive.\\
The most commonly described physiological mechanisms causing a tree to die are hydraulic failure and carbon starvation \citep{sevanto2014}. Hydraulic failure describes this desiccation that might happen due to failed water transport \citep{mcdowell2011b}. Carbon starvation can be defined as the failure to maintain the metabolism or to defend against biotic agents because a negative carbohydrate balance is prolonged. These two mechanisms describing the plant vitality based on water transport and carbon assimilation, mobilization and concentration are expected to increasingly affect trees due to future extreme events \citep{brauning2017}. But also nutrient availability in the environment influences the survival of a tree since it affects the aboveground biomass allocation and can increase the water use efficiency and thus the risk of hydraulic failure and/or carbon starvation \citep{gessler2017}.\\
Tree mortality can be measured on different scales, by e.g. looking at the individual organism or at the plot level, where mortality is given as the percentage of dead trees. In this regard, a further differentiation based on species, angiosperms/gymnosperms is not made. Though, their coping mechanisms and thus their mortality rates against drought and stress differ  \citep[e.g.][]{chaparro2017}.\\
\cite{cailleret2017} discussed the reduced growth rates of forests before a mortality event occurs in about 84\% of the cases. Reduction of growth rate did not happen suddenly, but could start between 1 through 100 years before mortality. Differences were influenced by stress in different ways: type, duration, frequency, and intensity of stress factors predisposing and triggering mortality. This means drought-induced mortality, biotic-induced or mixed.\\
%especially on gymnosperms and drought induced mortality events (similar for drought together with biotic agents). but when biotic agents alone source of mortality: then shorter and more intense growth reduction (at p~\textless~0.1), with bark-beatle even more significant (p~\textless~0.05). longest and strongest growth reduction (median time for \(\Delta t_{m}~=~24~years\), annual growth ratio between dying and conspecific surviving trees at final stage, i.e. last year before mortality event \(g_{f,m}~=~0.29\)) found for other factors (not drought, not biotic agents; this includes interindividual competition)\\
%- species dying:\\
%like esh disease, directional, mortality only affecting specific species. can be observed from space, but due to spectral mixing and resolution mostly undetectable.\\
%from the ecological point of view very interesting, but restrained feasibility\\
The ability of a forest to remain its initial state is called \gls{resistence} \citep[\textit{cf.} glossary for definitions, too, e.g.][]{keersmaecker2014}. That is, environmental factors may change, but the forest still does not suffer from it. Disturbance also might occur with an impact on the forest, but not causing it to shift into an alternative state, because it shows high \gls{resilience}. \Gls{resilience} is the return rate into the equilibrium state. In this case, individual trees might die, but the system will return to its initial state. A broader idea to characterize an ecosystem based on its \gls{resistence} and \gls{resilience} is \gls{variance}.\\
\cite{dakos2014} describes resilience indicators from ecosystem time series. The idea is based on the fact that close to a \gls{tippoint} resilience is small. That is, a system with high resilience and thus far from a tipping point shows different behavior than a system with low resilience. This is mostly due to either one or several of the following mechanisms: rising 'memory', rising variability, and \gls{flickering}.\\
Natural ecosystems can shift from one stable state into another. That is, in one stable state the ecosystem will tend to move back to a state of stability (\gls{basinattraction}) during environmental changes. If the system approaches a \gls{tippoint}, then it might pe pushed over the boundaries and shift into an alternative stable state. The system then shows low resilience and low recovery rate. With this the system's response to environmental impacts is more pronounced, which appears as increased variance and autocorrelation in a time series \citep{dakos2014}. Before the moment of transition, also called local bifurcation point, the recovery rate and \gls{resilience} span an angle of eigenvalue $\lambda = \ang{0}$. This results in apparent big impact of small disturbances, because the system needs more time to dissipate. This phenomenon is called Critical Slowing Down (CSD) \citep{dakos2014}. However, not all systems react in the same way when approaching a transition. Some regime shifts lack CSD completely. This is for example the case for strong abrupt changes in environmental conditions. Therefore, it is important to have knowledge on the mechanisms behind a regime shift when identifying early warning metrics \citep{dakos2014}. In general, regime shifts can be characterized according to the driving mechanisms \citep{dakos2014}:
	\begin{itemize}
		\item Slow environmental change towards a \gls{tippoint}
		\item Slow-fast cyclic transitions
		\item Stochastic resonance
		\item Noise-induced transitions
		\item Long transient upon extreme events
		\item Big stepwise changes in external conditions
	\end{itemize}
	
Not all regime shifts are announced in CSD, nor do they necessarily show tipping points \citep{dakos2014}. Slowly changing drivers to tipping points, slow-fast cyclic transition, and in cases also stochastic resonance and noise-induced transitions exhibit CSD as can be seen in figure \ref{fig:CSD_indics}. False Positives occur when underlying stochastic regime of perturbations changes. CSD is not only sensitive to system's dynamics, but also to the environmental dynamics.
	
\begin{figure}
	\centering
	\includegraphics[width = 0.9\textwidth]{CSD_indicators_Dakosetal2014}
	\caption{Critical Slowing Down indicators (e.g. standard deviation \textit{SD} and temporal autocorrelation at lag-1) can show transitions in bistable systems (a), but also in a system without alternative state with (b) or without (c) tipping point. Still, CSD can be observed \citep{dakos2014}.}\label{fig:CSD_indics}
\end{figure}	

%\cite{dakos2012} implies two major challenges for detecting leading indicators: high-frequency sampling or experiments, and the lack of a clear framework. The former can be overcome with satellite data, as high frequency sampling is done by several satellites: the proposed MODIS NDVI data feature a revisit time of 1 day although not every observation can be used due to cloud coverage and other quality reducing factors. The latter is given in \cite{dakos2014}.\\


%%% Remote Sensing & Dendrochronology for Forest Resilience

%satellite data: information on photosenthetic activity of forests (vegetation indices)
%question: would NDVI based TAC as a proxy of resilience work better on other biomes than the tropics? NDVI satureates at high LAI, only sees the crown, especially when so complex and multilayered as the tropics, where other factors, like shadowing, lokking angle and others highly influence the signal \citep{morton2016} \\
%role of detrending? detrending necessary to observe the small changes that are not due to seasonal growing and phenology, but are due to perturbations. detrending tries to fit a model (e.g. sinusoidal) to the time series. amplitude will refer to maximum phenological difference, phase to the length of the season. in the mediterranean the seasonality is far more pronounces and also more variable, making it more complicated to describe. this would require a more flexible seasonality modelling (detrending) approach.
%tree rings: provide "complete" archive of one data point per year (tree ring); ring width depending on the environmental conditions within the growing season. depending on changes in environmental conditions and their time of occurrence within the season, they might influence the year of occurrence or the following. 
%when using tree rings for extraction of information on the forest stand, three major points need attention: IADF (Intra-Annual Density Fluctuations, see above); spatial and temporal resolution of tree-ring DB; homogeneity of the forest stand. says Mathieu. reference in literature. - to do.