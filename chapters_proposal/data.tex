\subsection{Data}\label{subsec:data}
%The following chapter will describe the data that will be used for deriving Early Warning Signals (EWS) for the prediction of an approaching tipping point. Remote Sensing data for this study all come from spaceborne sensors and can be divided by their acquisition mechanism into optical and radar instruments. They are provided as imagery, hence as raster products. The third data source for EWS is dendrochronology data, that is tree rings, and will be treated in a separate third chapter within this section. They are given for an according study area and are thus treated as vector data.
% short explanation on multispectral satellite data: acquisition, processing steps, EMS and vegetation, chances and draw-backs when using MS data
% give formulas for derivation
% short explanation on SAR, how VOD is derived, characteristics of SAR signal for vegetation, chances and drawbacks

% Spatial & Temporal Resolution!!!
\subsubsection{Sensors}
By now, there is a variety of optical sensors in space with differing spatial and temporal resolution that fit the purpose of vegetation monitoring. Common pixel sizes vary between 1.5~m (e.g. SPOT Végétation) and several hundreds to thousands of meters (e.g. MODIS, MERIS, Sentinel-5 Tropomi). The revisit time of these sensors mostly depends on the number of satellites supporting the fleet. Recently launched Sentinel-2 operates with two identical satellites to provide a temporal resolution of 5 days at the equator and 2-3 days at mid-latitudes. For the study at hand, both the spatial resolution and the length of the archive matter most, since the number of data points (time steps) for each time series should still be given with a high revisit time to ensure sufficient cloud-free observations and long archive. Therefore, the high-resolution sensor LANDSAT~TM was chosen, since it provides an archive since 1972 and high enough ground resolution to capture the small scale forests of Catalonia.\\
%\subsubsection{Optical Data}\label{subsubsec:optical}
Optical sensors refer to camera-like sensors. The technique describes the capturing of a range of wavelength in the electromagnetic spectrum (i.e. a band), therefore also called multispectral (multiple spectra). This wavelength range can represent the visible light, near-infrared (NIR) or short-wave infrared (SWIR). Since the light source is mostly the sun, these are called passive instruments: they only capture light, but do not emit any of the received energy. %The energy is captured after it travels through space, transmits the atmosphere in varying proportions, hits the surface where it gets reflected, then travels back through the atmosphere and eventually hits the sensor. 
The amount of reflected energy depends on the albedo of a surface and is unique to each surface. Hence is it theoretically possible to classify and characterize each surface based on its reflectivity.
These processes of transmission, reflection, and absorption already feature the importance of knowing the amount of energy emitted by the sun as well as the given processes in the atmosphere to remodel the according reflectance of a given pixel. Therefore, optical remote sensing data can be classified into different processing levels. The datasets at hand will either be processed until Level~3 (ready-to-use products: including georeference) or delivered as Level~1C (atmospheric corrected and georeferenced individual bands) on which simple calculations will be performed to obtain vegetation indices (then Level~3).\\

\newpage
\subsubsection{Indices}
The vegetation indices will aim to depict different ground processes that are expected to exhibit CSD. These processes are a representation of plant vitality and/or leaf properties. The first index is called Normalized Difference Vegetation Index (NDVI). NDVI is mostly sensitive to photosynthetic activity. That is because Chlorophyll-a and chlorophyll-b absorb light in the red and green spectrum, but highly reflect in NIR. Typical vegetation spectra with varying chlorophyll contents can be seen in figure \ref{fig:vegspec}.

\begin{figure}[htp]
	\centering
	\begin{subfigure}{0.49\textwidth}	
		\centering
		\includegraphics[width = 0.95\textwidth]{veg_spectra_CAB}
	\end{subfigure}
	\begin{subfigure}{0.49\textwidth}	
		\centering
		\includegraphics[width = 0.95\textwidth]{veg_spectra_CW}
	\end{subfigure}
	\caption{Typical vegetation spectra with LAI = 4, varying chlorophyll and leaf water content modelled in SLC demo. Higher chlorophyll content lowers the reflectance in the visible light, whereas leaf water content influences the absorption of light in higher wavelengths.}\label{fig:vegspec}
\end{figure}	

NDVI is calculated by dividing the difference of the reflectance of these bands over their sum \citep[\textit{cf.} formula \ref{eq:ndvi}, after][]{tucker1979}. High photosynthetic activity is thus shown with a high NDVI. NDVI is widely known and widely used due to its easy calculation and relative robustness against varying atmospheric conditions.\\
% NDVI literature
% NDVI formula
\begin{equation}\label{eq:ndvi}
	NDVI = \frac{R\textsubscript{NIR} - R\textsubscript{Red}}{R\textsubscript{NIR} + R\textsubscript{Red}} 
\end{equation}\\

The Enhanced Vegetation Index (EVI) works similar to the NDVI, but accounts for high LAI-values. NDVI tends to saturate at high LAI values. By applying a gain factor to the entire fraction and accounting for canopy as well as aerosol background conditions, it overcomes this obstacle and is more sensitive to LAI and canopy architecture \citep{liu1995}. For most common sensors this is calculated as follows:\\
\begin{equation}\label{eq:evi}
	EVI = 2.5 \times \frac{R\textsubscript{NIR} - R\textsubscript{Red}}{R\textsubscript{NIR} + 6 \times R\textsubscript{Red} - 7.5 \times R\textsubscript{Blue} + 1}
\end{equation}\\

NBR compares the NIR with the SWIR water absorption feature at at 2.1~- 2.3~\textmu m \citep[\textit{cf.} formula \ref{eq:nbr}, after][]{garcia1991}. The NIR reflectance is mainly influenced by the LAI (leaf area index) and looking angle (BRDF). Photosynthetically active vegetation shows characteristic behavior in this wavelength region with very high reflectance compared to other surface types. It can reach up to 70\% reflectance in planophile plants with high LAI. The water absorption feature in the SWIR reflectance is mainly influenced by leaf water content (\textit{cf.} fig. \ref{fig:vegspec}) as well as LAI. Originally, the NBR was developed to show fire scars in vegetation, but since it is sensitive to changes in leaf water content, it could possibly depict CSD prior to a tipping point, too. Potentially the combination of EWS from NDVI and NBR could show additional value.\\

\begin{equation}\label{eq:nbr}
	NBR = \frac{R\textsubscript{NIR} - R\textsubscript{SWIR}}{R\textsubscript{NIR} + R\textsubscript{SWIR}} 
\end{equation}\\




%\subsubsection{Radar Data}\label{subsubsec:radar}
%Radar instruments capture energy in the microwave spectrum. Usually spaceborne radars are active instruments with a side-looking antenna. It emits polarized microwaves and detects the amplitude and phase of the backscattered wave. This allows for substantially different analyses than optical instruments. As a rule of thumb, electromagnetic waves interact with objects and structures of size of at least their own wavelength. As such, microwaves transmit the atmosphere for their biggest part, not like optical data that is highly influenced by cloud coverage.\\
%Secondly, since the phase of the wave can be measured, this technology allows for analysis of polarization. This comes in handy for vegetation monitoring. Vegetation is the main source of depolarization. As the microwave hits, for instance, a forest the wave will partially penetrate and interact with either branches or the trunk - depending on the wavelength (frequency). Due to depolarization and socalled volume-scattering microwaves are sensitive to biomass(literature).\\
%
%
%% proper placement of the below paragraph
%The above described data sources and different indices are all sensitive to different leaf properties: NDVI and EVI are sensitive to chlorophyll content and thus to photosynthetic activity, NBR to leaf water content, and VOD mostly to biomass. Therefore, they cover a wide range of parameters where CSD can be expected. The main difference between NDVI and EVI is that NDVI saturates at high LAI-values. However, it is very easy to calculate and widly used and known. I would expect EVI to be more sensitive to CSD, but the user community might be more susceptible to implement NDVI-based EWS since familiarity is given.
	

%\subsubsection{Dendrochronology Database}\label{subsubsec:dendrodb}

%a) MODIS NDVI, LAI, EVI, NBR? AMSR-E VOD? MODIS on Terra \& Aqua satellites: time series of NDVI multispectral data since 1999 offers long-term time series with medium resolution (200~m) for earth surface monitoring\\
%b) Landsat NDVI, EVI, since 1972\\
%Level 3, products. explanation of pre-processing and product generation\\
%optical/multispectral data, radar data\\

