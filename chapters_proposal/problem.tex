\section{Problem Definition}\label{sec:problem}
Remote Sensing data offer the possibility to monitor on a large spatial and temporal scale. Using this asset, there have been studies on the impacts of tree mortality using remote sensing \citep[e.g.][]{breshears2005, meigs2011}, or analyzing extent and patterns after biotic-induced mortality events \citep[e.g.][]{meddens2012, wulder2006}. However, this mostly addresses large areas that die off simultaneously and does not look at the amount of mortality within a stand. This does not need to result in a die-off and can occur due to a multitude of causes \citep{cailleret2017, gessler2017, hartmann2015, mcdowell2011a, mcdowell2013, seidl2011}. Remotely sensing based resilience monitoring methods have been demonstrated have shown the potential for resilience monitoring of forests as a basis for early warning \citep{keersmaecker2014, verbesselt2016}. However, this has not been tested for tree mortality monitoring.\\
Therefore, this thesis will look at the ability to predict tree mortality with EWS. This will be performed based on optical Landsat~TM satellite imagery from which vegetation indices will be derived for the time span of the start of the Landsat series in 1982 until 2016. From these time series generic and spatial EWS will be derived and used as explanatory variables to predict the amount of tree mortality within given plots in a study area in Nothern Spain. The performance of the model will be tested independently. This approach will not focus on short-term external disturbances, such as windthrow, fire or flooding, but will be seeking to reconstruct the mortality in stress-induced forest ecosystems that is usually preceeded by changes in tree function as well as in structure \citep{hartmann2015}.\\
Since tree mortality is usually announced in reduction of growth rate and is thus a process that includes a temporal component, it is promising to assess the predictability from multitemporal data related to forest resilience rather than try to predict it from a single time step. This research is thus aiming to predict tree mortality on a plot level using EWS as predictors from remote sensing time series. This research will thus assess the performance of remotely sensed resilience indicators (EWS) for the predictability of tree mortality in a Mediterranean ecosystem.\\