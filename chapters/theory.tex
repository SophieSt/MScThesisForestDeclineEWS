\section{Theory on Tree Mortality, Forest Decline, and Early Warnings}\label{sec:theory}
Forest decline can be defined as tree mortality affecting an entire forest stand \citep{martinez2012}, but also leaf discoloration and leaf loss in a large extent can qualify as forest decline \citep{chaparro2017}. Either way, both definitions imply a large and obvious reaction of the forest indicating that a forest is massively affected. This thesis will focus on the latter where a forest is defined as declined if an area larger than 0.03~km\textsuperscript{2} shows tree mortality larger than 5\% or is affected by more larger than~50\% in the sum of both leaf loss and leaf discoloration in the canopies of at least one abundant tree species with a canopy cover larger than 15\% \citep{chaparro2017}.\\
Both the processes causing tree mortality and the exact definition of it remain a subject of research \citep{hartmann2015}. From a theoretical point of view tree death occurs at the point of minimum vitality \citep{dobbertin2005, grivcar2012}. From an anatomical and developmental perspective this point is less clear \citep{schenk2008}: one or more organs and tissues of the tree can for instance fatally desiccate when at the same time, other tissues like apical, cambial, and/or root meristems might still be intact and keep a plant alive.\\
The most commonly described physiological mechanisms causing a tree to die are hydraulic failure and carbon starvation \citep{sevanto2014}. Hydraulic failure describes this desiccation that might happen due to failed water transport \citep{mcdowell2011b}. Carbon starvation can be defined as the failure to maintain the metabolism or to defend against biotic agents when a negative carbohydrate balance is prolonged. These two mechanisms describing the plant vitality based on water transport and carbon assimilation, mobilization and concentration are expected to increasingly affect trees due to future extreme events \citep{brauning2017}. But also nutrient availability in the environment influences the survival of a tree since it affects the above-ground biomass allocation and can increase the water use efficiency and thus the risk of hydraulic failure and/or carbon starvation \citep{gessler2017}.\\
Tree mortality can be measured on different scales, e.g. by looking at the individual organism or at the plot level, where mortality is given as the percentage of dead trees. In this regard, a further differentiation based on species, angiosperms/gymnosperms is only applied in the form of accounting for species. Though, their coping mechanisms and thus their mortality rates against drought and stress differ  \citep[e.g.][]{chaparro2017}.\\
%especially on gymnosperms and drought induced mortality events (similar for drought together with biotic agents). but when biotic agents alone source of mortality: then shorter and more intense growth reduction (at p~\textless~0.1), with bark-beatle even more significant (p~\textless~0.05). longest and strongest growth reduction (median time for \(\Delta t_{m}~=~24~years\), annual growth ratio between dying and conspecific surviving trees at final stage, i.e. last year before mortality event \(g_{f,m}~=~0.29\)) found for other factors (not drought, not biotic agents; this includes interindividual competition)\\
%- species dying:\\
%like esh disease, directional, mortality only affecting specific species. can be observed from space, but due to spectral mixing and resolution mostly undetectable.\\
%from the ecological point of view very interesting, but restrained feasibility\\
The ability of a forest to remain in its initial state is called \gls{resistance} \citep[compare glossary for definitions, e.g.][]{keersmaecker2014}, which in this case means that the tree cover remains despite a distortion. That is, environmental factors may change, but the forest still does not suffer from it. Disturbance also might occur with an impact on the forest, but not causing it to shift into an alternative state, because it shows high \gls{resilience}. \Gls{resilience} is the return rate into the equilibrium state. In this case, individual trees might die, but the system will return to its initial state. \cite{dakos2014} describes resilience indicators from ecosystem time series. The idea is based on the fact that close to a \gls{tippoint} resilience is small. That is, a system with high resilience responds faster to perturbations, which is manifested as rising memory, variability and increased \gls{flickering}.\\
Natural ecosystems can shift from one stable state into another when a particularly heavy perturbation hits the system. This shift is called \gls{tippoint}. Other ecosystems have only one stable state and will recover from perturbations, or even reassemble without shifting to a qualitatively different state. That is, normally an ecosystem will tend to move back to a state of stability (\gls{basinattraction}) during environmental changes. If the system's resilience is low, it can be pushed over the boundaries and shift into an alternative stable state. Before this moment, the system exposes low resilience and a low recovery rate. With this the system's response to environmental impacts is more pronounced, which appears as increased variance and autocorrelation, which makes it observable in time series \citep{dakos2014}. Before the moment of transition, also called local bifurcation point, the recovery rate and \gls{resilience} span an angle of eigenvalue $\lambda = \ang{0}$. This results in apparent big impact of small disturbances, because the system needs more time to dissipate. This phenomenon is called Critical Slowing Down (CSD) \citep{dakos2014}. However, not all systems react in the same way when approaching a transition. Some regime shifts lack CSD completely. This is for example the case for strong abrupt changes in environmental conditions. But also less stable ecosystems that will not shift to another state can expose CSD.\\
\cite{dakos2012} implied two major challenges for detecting leading indicators: high-frequency sampling or experiments, and the lack of a clear framework. The former can be overcome with satellite data, as high frequency sampling is done by several satellites: the proposed MODIS NDVI data feature a revisit time of 1 day although not every observation can be used due to cloud coverage and other quality reducing factors. The clear framework of extracting leading resilience indicators is proposed in \cite{dakos2014}, who also launched a website for easy and low-threshold access at \url{http://www.early-warning-signals.org/home/}, where the framework, logical steps and theory is explained in simpler terms and with additional material.\\
