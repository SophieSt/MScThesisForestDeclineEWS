% conclusion
\section{Conclusions}\label{concl}
In the following section the research questions will be answered based on the previously discussed results.

\begin{itemize}
\item {Has there been Critical Slowing Down (CSD) prior to the forest decline event in Catalonia in 2012 which was captured in satellite time series?}\\
The most sensitive parameter in the extraction of Critical Slowing Down was found to be the size of the rolling window in which the EWS were calculated. The typical rolling window length of half the length of the time series showed the highest robustness for extracting a slowing down trend. The detection of Critical Slowing Down was mostly insensitive to the smoothness of the detrending. 

\item{If so, is the forest decline linked to the reduction in stability of the ecosystem?}\\
Yes, the reduction of residual deviance due to extending the null model with EWS showed that resilience indicators help explain this. NDVI-based EWS-indicators helped explain almost 4\% of the deviance in the data. Given that the model only explained 33\% of the total deviance, the improvement is statistically relevant. The EWS-based indicators were mostly significant at p~$<$~0.001, emphasizing the role of resilience in forest decline. 

\item{Which EWS are most sensitive to explaining forest decline?}\\
No single individual EWS was found to be particularly indicative of forest decline. The model explaining the highest amount of deviation was the one using the first six Principal Components of the NDVI-based EWS. The single individual indicator explaining most deviance was the trend in spatial variance in NDMI. Generally, the EWS extracted from NDVI were outperforming NDMI and EVI in terms of model fit and model performance on an independent test dataset. EVI showed overall the least explanatory power. The weakness of NDMI and EVI compared to NDVI is attributed to higher Signal-to-Noise Ratio for the bands used for deriving NDVI.
\end{itemize}
