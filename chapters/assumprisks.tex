\section{Feasibility}\label{sec:feasibility}
The following points have been identified as risks and potential complications for me to carry out this research:
\begin{itemize}
	\item The extraction of the EWS might be intense in computation power and thus in time. Particularly the recalculation with different window sizes is what will make the extraction tedious. It is yet to be determined where to do the extraction. One possible solution could be to perform this on a server or to parallelize processes.
	\item Different mechanisms can lead to regime shifts and different mechanisms can show up in the EWS. Wishful would be not only to predict tree mortality with a flexible and accurate model, but to predict it in a way that it supports further understanding of mechanisms. Inference is not always straightforward for flexible models.
	\item Reference data come from CREAF research center in Barcelona. Communication and if necessary the access to supporting information and data might take some extra time.
	\item Researchers at CREAF have once tried to predict mortality from space, but did not succeed to receive meaningful results. Potentially, the ability to predict with remote sensing data is limited. The potential to use time series however might help to take a step further from the simple monotemporal extraction of predictors.
	\item The main supervisor (J. Verbesselt) is particularly busy and ambitious. This requires planning joint meetings long time ahead and complicates the process of decision-taking where needed. The second supervisor (U. Sass-Klaassen) will support this thesis on its interdisciplinary nature for the intepretation and discussion of results. Together, this helps to profit from their networks as well as from the wide range of knowledge brought into supervision.
\end{itemize}