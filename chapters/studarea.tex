\subsection{Study Area}\label{subsec:studarea}
The Mediterranean is a vulnerable ecosystem that could be highly affected by climate change \citep{anav2011} with threats to the regional terrestrial carbon cycle and the vegetation dynamics. Climate change in the region is associated with higher temperatures, less rainfall and thus more drought and water stress for plants in the region. Catalonia was hit by several droughts in the past decades. 2004-2008, 2012 and 2016 were years with intense drought that affected large stands of the forested surface \citep{chaparro2017}. The forest decline of the drought-year 2012 was particularly intense in terms of forest decline occurrence. \cite{chaparro2017} already researched the environmental drivers of the declines, making it an ideal study area to assess the role of resilience in the occurrence of forest decline.\\
The study area included the entire forested surface of the region of Catalonia (Autonomous Community of Catalonia/Republic of Catalonia), which sums up to roughly 13000~km\textsuperscript{2} or 40\% of Catalonia's terrestrial surface \citep{chaparro2017}. Catalonia is widely forested and offers a variety of landscapes and climates due to the presence of the Pyrenees in the North and the proximity to the Mediterranean Sea in the Southeast and the Central Depressions and coastal mountain ranges. Climatically it is classified into Alpine climate in the Pyrenees, Maritime or Oceanic in the valleys and Mediterranean on the coast and the inland. It is thus diverse and features different kinds of climatic and environmental conditions for the vegetation. It features not only large areas of forest at the boundary of two biogeographic regions (Mediterranean and Euro-Siberian) but also a variety in climatic variables such as mean annual rainfall and temperature, as well as their annual distribution, but also different species. The most frequent tree species are \textit{Pinus halepensis} covering 2430~km\textsuperscript{2} and \textit{Quercus ilex} with 2000~km\textsuperscript{2} \citep[\citeauthor{chaparro2017}, \citeyear{chaparro2017} after][]{creafsol}.\\
The DEBOSCAT network in Catalonia is a unique survey for the assessment of forest health with field data describing all forested areas $>$ 0.03~km\textsuperscript{2} that were affected by forest decline. Small forest patches affected by forest decline were not included in the analysis, that is False Negatives of less than 0.03~km\textsuperscript{2} might occur and bias the result as they might indeed show CSD. Although droughts are common in the region, the 2012 drought was characterized by intense temperature anomalies affecting broadleaved species in particular \citep{chaparro2017}. It was preceded by a drought that lasted from October 2004 until October 2008, which might have already weakened the abundant forests and lowered their resilience. Most of the declined forests are situated inland. A cluster with many declined plots shows up northeast of Vic, a smaller one around Solsona and some more around the Central Depression. For a detailed map of the declined plots see figure \ref{studarea}.


\begin{figure}[H]
	\centering
	\makebox[\textwidth][c]{\includegraphics[trim = 0 20 0 10, clip, width = 1.2\textwidth]{DeclineMask2012.png}}
	\caption{Overview of the forest extent in Catalonia, declined forests are marked as grey polygons.}\label{studarea}
\end{figure}

