\section{NDVI-based EWS maps}
Maps of Kendall's $ \tau $ of the EWS from MODIS NDVI within the study area. Grey polygons show the decline areas of 2012. Red colors show a positive trend, blue colors a negative trend.\\

\begin{figure}[htpb]
	\centering
	\includegraphics[trim = 55 55 10 10, clip, width = 0.85\textwidth]{NDVIar1Tau.png}
\end{figure}	

\begin{figure}[htpb]
	\centering
	\includegraphics[trim = 55 55 10 10, clip, width = 0.85\textwidth]{NDVIdensratTau.png}
\end{figure}	

\begin{figure}[htpb]
	\centering
	\includegraphics[trim = 55 55 10 10, clip, width = 0.85\textwidth]{NDVIsdTau.png}
\end{figure}	

\begin{figure}[htpb]
	\centering
	\includegraphics[trim = 55 55 10 10, clip, width = 0.85\textwidth]{NDVIskTau.png}
\end{figure}	

\begin{figure}[htpb]
	\centering
	\includegraphics[trim = 55 55 10 10, clip, width = 0.85\textwidth]{NDVIkurtTau.png}
\end{figure}	

\begin{figure}[htpb]
	\centering
	\includegraphics[trim = 55 55 10 10, clip, width = 0.85\textwidth]{NDVIspVarTau.png}
\end{figure}	

\begin{figure}[htpb]
	\centering
	\includegraphics[trim = 55 55 10 10, clip, width = 0.85\textwidth]{NDVIspSkewTau.png}
\end{figure}	
