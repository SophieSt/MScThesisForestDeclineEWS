\section{Interaction Effects of Model Coefficients}
\subsection{NDVI-based EWS Coefficients}
\begin{table}[H]
	\centering
	\makebox[\textwidth][c]{\includegraphics[trim = 0 100 0 10, clip, width = 0.9\textwidth]{coeffNDVI.pdf}}
	\caption{Coefficients of NDVI-based EWS for the two most abundant species in the study area \textit{Pinus halepensis} and \textit{Quercus ilex}. The base case scenario refers to the first class of species which is \textit{deciduous Quercus} and is reported here for the sake of completeness. Significance levels are reported per species. Positive coefficients indicate a positive relationship between EWS and forest decline.}\label{app:coeffNDVI}
\end{table}

\subsection{NDMI-based EWS Coefficients}
\begin{table}[H]
	\centering
	\makebox[\textwidth][c]{\includegraphics[trim = 0 100 0 10, clip, width = 0.9\textwidth]{coeffNDMI.pdf}}
	\caption{Coefficients of NDMI-based EWS for the two most abundant species in the study area \textit{Pinus halepensis} and \textit{Quercus ilex}. The base case scenario refers to the first class of species which is \textit{deciduous Quercus} and is reported here for the sake of completeness. Significance levels are reported per species. Positive coefficients indicate a positive relationship between EWS and forest decline.}\label{app:coeffNDMI}
\end{table}

\subsection{EVI-based EWS Coefficients}
\begin{table}[H]
	\centering
	\makebox[\textwidth][c]{\includegraphics[trim = 0 100 0 10, clip, width = 0.9\textwidth]{coeffEVI.pdf}}
	\caption{Coefficients of EVI-based EWS for the two most abundant species in the study area \textit{Pinus halepensis} and \textit{Quercus ilex}. The base case scenario refers to the first class of species which is \textit{deciduous Quercus} and is reported here for the sake of completeness. Significance levels are reported per species. Positive coefficients indicate a positive relationship between EWS and forest decline.}\label{app:coeffEVI}
\end{table}