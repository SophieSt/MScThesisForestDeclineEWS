\section{Problem Definition}\label{sec:problem}
According to the definition of the Food and Agriculture Organization of the United Nations (FAO), forest covers around 30\% of the earth's land surface \citepalias{unepForest} and their contribution to the terrestrial gross primary production of biomass is estimated at 75\% \citep{beer2010}. They take up atmospheric carbon and store it in their  biomass above and below the ground. But rising temperatures, changes in rainfall patterns and accelerating rates of climate change leave forests vulnerable to decline \citep{choat2012}. The physiological mechanisms behind forest decline are still not completely clear \citep{sala2010}, but also the explanation of the spatial occurrence is still pending \citep{chaparro2017}, so the question of 'what features are associated with forest decline?' is still not sufficiently answered.\\
Remote Sensing data offer the possibility to monitor on a large spatial and temporal scale. Using this asset, there have been studies on the impacts of tree mortality using remote sensing \citep[e.g.][]{breshears2005, meigs2011}, or analyzing extent and patterns after biotic-induced mortality events \citep[e.g.][]{meddens2012, wulder2006}. Remote sensing based resilience monitoring methods have demonstrated the potential to describe forest resilience on a large scale and its relationship with environmental factors \citep{keersmaecker2014, verbesselt2016}.\\
This thesis will look at the ability to predict forest decline with remotely sensed resilience indicators. This will be performed based on optical MODIS satellite imagery from which vegetation indices will be derived for the time span of the start of the MODIS mission in 2000, until the end of the growing season preceding a large drought-induced decline event that took place in Catalonia, northern Spain in 2012. From these time series generic temporal and spatial EWS will be derived and used as explanatory variables to model forest decline in the study area (Catalonia). The performance of the model will be tested independently. This approach will not focus on short-term external disturbances, such as windthrow, fire, or flooding, but will be seeking to reconstruct resilience in stress-induced forest ecosystems. Resilience indicators can be extracted because such dramatic events like forest decline are usually preceeded by changes in tree function as well as in structure \citep{hartmann2015}.\\
Since tree mortality is usually announced in reduction of growth rate and is thus a process that includes a temporal component, it is promising to assess the predictability from multitemporal data related to forest resilience, rather than trying to predict it from a snapshot (thus one single time step). This research therefore aims to predict forest decline on a grid-cell level of available satellite products using EWS as predictors from remote sensing time series. This thesis will assess the performance of remotely sensed resilience indicators (EWS) for the predictability of forest decline in a drought-stressed Mediterranean ecosystem.\\
