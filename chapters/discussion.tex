% discussion
\section{Discussion}\label{discussion}
The following chapter will explain the previously stated results towards their importance in answering the previously stated research question, relates them to other publications and will provide insight into the limitations of this research. Besides that, it will also point out findings that require future research. It starts by the most important findings and will then move towards the interpretations and open questions.\\

\subsection{Early Warning Signals as Predictors of Forest Decline}\label{disc:pred}
The overall question that led to this research, is whether Early Warning Signals can provide an estimate of the forest ecosystem's stability in order to improve the predictability of forest decline. This question arises from two aspects: 1) can we associate possible dependencies, relationships and potential causes with forest decline and 2) how can we better predict forest decline in the future and thus also better monitor the risk? The former question seeks insights into the mechanisms, the latter focuses on model performance.\\
Adding resilience indicators as a measure of the stability of the ecosystem significantly to modelling forest decline from otherwise environmental features significantly reduced the remaining deviance in the decline dataset. That is, it helps explain forest decline. Besides this, it also contributed to a better model performance. Resilience indicators managed to explain up to 4\% more of the total deviance compared to the null model of \cite{chaparro2017}. The significance level of the best predictors was high (p~$<$~0.001), indicating a clear relationship of forest decline with stability of the ecosystem. This suggests that resilience plays an important role in understanding why a certain forest declines and also significantly improves the ability to model forest decline. The improved predictability was found with all three vegetation indices, although NDVI outperformed NDMI and EVI.\\
All models still failed to explain the spatial occurrence of forest decline. This was indicated by low sensitivity values towards True Positives across all models. The explained deviance was still around 30\% which suggests that forest decline is highly driven by other parameters or non-linear combinations of those and/or the given predictors. However, the null model as described in \cite{chaparro2017} explained 40\% of the deviance, whereas the null model as set up in this research only explained 27\%. This difference can be explained by the upscaling approach. \cite{chaparro2017} encountered decline-pixels with more than one abundant species which they treated as two observations. In the approach at hand, only the first observation per grid cell (pixel) was used for easier data handling. This might have reduced the variability of observations in some species that did not feature many declined areas anyway, further worsening the problem of unbalanced priors. The inability to describe the complete feature space might then have resulted in the worse model fit.\\
Critical Slowing Down was found in all EWS. Their spatial pattern was similar over the study area among indicators - especially among indicators depicting it in a similar way, like the spectral indices - though not spatially consistent. Indicators with similar spatial pattern were ACF(1), AR(1), and the density ratio. Since they all aim to identify the impact of the previous observation on the subsequent one \citep{dakos2012}, this was to be expected.\\
The main finding that remotely sensed resilience plays an important role in explaining forest decline is supported by other findings. \cite{verbesselt2016} conducted an analysis on remotely sensed resilience in tropical forests which showed that resilience can be derived from time series of optical satellites, which provide consistent archives of resilience indications worldwide for up until 30 years ago. \cite{cailleret2017} synthesized that growth rate decreased prior to mortality events. This suggests that changes in the stochastic regime appear prior to a mortality event and can thus also happen prior to forest decline. \cite{rogers2018} conducted a research to assess the relationship between remotely sensed EWS and tree mortality and found significant differences among some EWS, like AR(1), density ratio, kurtosis, and conditional variance, between the control group and sites with higher mortality, also suggesting that EWS from satellite data can be used to predict forest decline. They also found that the relationships were not consistent or significant for the multitude of other EWS. This also agrees with the findings at hand, that some EWS like spectral indices (AR(1) or ACF(1)) show more explanatory power than e.g. skewness. \\


\subsection{Data Quality}
The above described reduced observation density towards the end of the years follows the Mediterranean climate. Mediterranean climate is characterized by dry, hot summers and mild, wet winters and with this the precipitation maximum in fall and winter \citep{lionello2006}. The clouds that generate rain and snow block the view of the satellite, which results in missing ground observations. Therefore, the seasonality pattern seen in the histogram of missing values, shows that the expectation that the density of missing values is higher in fall and winter. The peak in the beginning of the time series (end of 2000) that affected a large area of Eastern Catalonia and follows a clear boundary to its west that does not coincide with any geographic features, but rather with the viewing angle of the satellite and is thus attributed to sensor malfunctioning or recalibration that can still be undertaken in the beginning of a satellite mission.\\
Overall, the data quality in terms of completeness is high. Even after filtering with the provided quality flags and outlier removal, more than 97\% of the observations remained in most time steps. This is more than could be expected from for a single satellite overpass per observation. However, the MODIS 16-day vegetation index product is a composite of 16 daily observations \citep{huete2002} and can thus deliver a higher quality through reduced temporal resolution. Filtering for outliers outside the range of normally occurring values made sure that the extraction of EWS was not skewed. Since the EWS aim to depict e.g. rising variability in the time series, these outliers would result in an underestimation of the underlying resilience pattern. Therefore, the step of outlier removal is of high importance. Uncertainty of the signal due to changing atmospheric conditions like aerosol composition or haze even after atmospheric correction is still possible. This might be reflected in the seemingly noisy EWS maps where moderate trends with large differences even in neighboring pixels were found. Further research is needed to identify the origin of this seemingly high noise level. A high noise level might as well obscure the trend in EWS.\\



\subsection{Sensitivity Analysis}\label{disc:csd}
The sensitivity analysis showed that the detection of Critical Slowing Down (CSD) was sensitive to the size of the rolling window within they were calculated and from within which the trend statistic was calculated. \cite{dakos2008} suggested to assess the relationship of the trend statistic with the parameter setting based on areas that show consistent trends in the sensitivty plots. They used time series of climate indicators that showed CSD prior to drastic climatic shifts, like the end of the big last glaciations or the Saharan vegetation. Their time series showed clear demarcable areas in which the parameter setting did not allow to robustly detect the ongoing trend and areas in which the trend was clear and persistent. The time series at hand that were used in the sensitivity analysis featured demarcable regions as well, but these were less clear. The difficulty of this research, though, was to summarize the outcome of the sensitivity analysis for the entire region of Catalonia, too. Within the sensitivity heatmaps that show the magnitude of trend depending on rolling window size and detrending flexibility, some pixels showed clear hotspots of high trends for detrending with a small bandwidth and a rolling window size of about half the length of the time series. Others were less clear or showed several of such hotspots. \cite{dakos2008} were able to analyze one time series per event. In this research the dataset was too big and the visual assessment of each time series was not feasible. Therefore, the resilience indicators needed to be extracted in an automatic way.\\
The optimal parameter setting for the Catalonian forests might vary depending on several other environmental or physiological factors that could affect the response time of a forest and by the size of the rolling window in which they would be detectable. These could include species, ground water level, or structural forest type. Further research is needed to identify and describe the effect of such conditions on the rolling window size.\\
The chosen bandwidth of four for trend filtering with a Gaussian kernel was supported by the distribution of the residuals which mostly did not feature seasonality anymore. The occasionally remaining seasonality could have been further removed, but would have come to the risk of following the data too closely in other pixels \citep{dakos2012}. This kind of overfitting would have again resulted in an underestimation of the trend, particularly in areas with weaker seasonality and in which the growing season and vegetation activity follows the current weather conditions more closely. Applying such a flexible filtering would have removed a trend in variability that the EWS aims to depict. Further research is needed on the automatic selection of an appropriate detrending method.\\
The size of the rolling window was found to be optimal at around 100-150 time steps which encompasses approximately half the length of the time series. Publications on the theory of EWS typically used this rolling window size as well after conducting a sensitivity analysis \citep{dakos2008, dakos2012} which led the authors of the R package \textit{earlywarnings} \citep{earlywarnr} to state on their according \href{http://www.early-warning-signals.org/time-series-methods/metric-based-indicators/general-steps-for-rolling-window-metrics/}{EWS-website} that half the length of the time series is typically used in extracting EWS from time series. In this regard, this research fits with previous findings.\\
It should also be noted, that the sensitivity analysis was only conducted on NDVI-based ACF(1) due to feasibility and time constraints. An assumption for this is that the response time for all indicators and especially all vegetation indices is comparable. The memory effect in the time series might or might not be of longer duration in e.g. the leaf water content sensitive NDMI time series. Future research should also examine if this assumption is valid. However, for the study at hand, this was out of scope due to time restraints. Therefore, the NDVI-based ACF(1) was used as a representative indicator.\\
The trend in the study area was visualized in maps which allowed for assessing the spatial homogeneity. Although the maps for each EWS showed clear spatial pattern, these patterns were widely lost when calculating the magnitude of trend in them (Kendall's $\tau$). The maps appeared to feature a high noise level especially for areas with moderate trend. That is, even areas within close proximity showed relatively high variation in the trend statistic. This suggests that the resilience indicators show a weakness due to high variance and thus result in being less specific. Consequently, moderate values of Kendall's $\tau$ should be treated with caution.\\


\subsection{Detection of Critical Slowing Down}
Critical Slowing Down is the phenomenon that an ecosystem becomes slower in returning to its initial state after a perturbation with decreasing resilience. This means that subsequent observations are increasingly correlated with each other. Hence, they then show a positive trend in temporal autocorrelation (ACF(1), AR(1) and density ratio). Positive trends were mostly found in Central and Eastern Catalonia coinciding partly with many decline plots that were found east of the city Vic. As such this is in agreement with the theory of CSD. The temporal autocorrelation in NDVI shows a positive trend for forest decline probability for the oak species \textit{Quercus ilex}. This agrees with the hypothesis that forests with lower resilience show higher temporal autocorrelation, thus they are more influenced by NDVI in previous time steps. However, the second most abundant species \textit{Pinus halepensis} does not show much association of forest decline with temporal autocorrelation ($\beta$ = -0.06 compared to 0.43 for \textit{Quercus ilex}). Given the large sample size of \textit{Pinus halepensis} plots, a random effect is highly unlikely. Further research is needed here as well, to investigate why some EWS are only sensitive for specific species and not for others. The trend is, however, very weak ($\beta$ = -0.06). The different coefficients of spatial variance can be explained such that in \textit{Quercus ilex} became increasingly spatially homogeneous towards the decline year, whereas \textit{Pinus halepensis} became more heterogeneous.\\


\subsection{Comparison among Vegetation Indices}\label{disc:vegind}
The models that used NDVI-based indicators performed better than those based on NDMI or EVI. Although EVI is calculated by accounting for atmospheric and soil conditions, it did not perform better than NDVI. The attempt to account for atmosphere might hinder the accurate capturing of the underlying ecosystem dynamics: atmospheric scattering mostly influences the smaller wavelength due to Rayleigh scattering, which is why the blue light reflectance is used in the calculation. However, due to this higher amount of scattering, the data quality is also reduced, adding uncertainty and variability to the signal. This might add a general factor of uncertainty to the EWS. Since the EWS aim to depict exactly the small variability in the data, the addition of the blue light reflection might obscure the patterns. Similarly, NDMI performed slightly less than NDVI. NDVI uses Near Infrared (NIR) and Red reflectance to derive an indication of photosynthetic activity, whereas NDMI uses NIR and Short-Wave Infrared (SWIR) to derive an indication of leaf water content. Both NIR and Red light are close to the sun's emissivity maximum. Since reflectance is a fraction of received to reflected energy, it depends on the amount of energy that reaches the Earth surface in the first place. The received energy originates mostly from the sun and shows a clear peak in the spectrum of the visible light. Towards larger wavelengths this energy is far lower. Therefore, the reflectance level is more noisy. Sensors in larger wavelengths usually have a lower Signal-to-Noise-Ratio (SNR). In that way, a similar process might take place for EVI and NDMI. The increased uncertainty in at-sensor signal might obscure the increasing variability that the EWS aim to depict.\\
Both the spatial pattern in the maps of the EWS-trends and the pattern in the eigenvalues of the EWS (PCA) show that the vegetation indices depict the same slowing down on the ground. The spatial variation in each of them is similar. This follows from similar spatial patterns (compare figure \ref{res:ndvi_ews_maps} and EWS maps for NDMI and EVI in Appendix A) and also from the similar contribution of each of the EWS for each Principal Component. It was followed that the indication of resilience extracted from each vegetation index is physiologically similar and not complementary. The remaining question in the comparison is thus which of the vegetation indices is most sensitive to CSD prior to forest decline. Given the higher amount of explained deviance and better model fit and performance, it was concluded that NDVI is most sensitive, followed closely NDMI.\\
EVI helps explain the deviance, but at a reduced level compared to the other two indicators. As explained above, the attempt to account for scattering also makes it subject to the increased level of noise that is induced exactly by Rayleigh-scattering in smaller wavelength regions. This also resulted in a smaller fraction of variation explained by the PCA. The first five Principal Components, that explained 89\% and 87\% of the variation in the NDVI and NDMI based EWS only explained 85\% of the variation in the EWS-dataset from EVI. Similarly, but less pronounced, this problem of increased variance also is visible for NDMI. The higher SNR of NDMI can therefore be held responsible for the slightly reduced explanatory power of its EWS. This is evenly apparent in the PCA, where each PC explains a smaller amount of variation in the data.\\
The similarity among vegetation indices towards their explanatory power was not expected. A study that focused on EWS in basal area increment (BAI), defoliation, and sapwood flow found that EWS differed among the indicator they were extracted from as well as among species \citep{camarero2015}. The study at hand looked at vegetation indices sensitive to photosynthetic activity as well as to leaf water content but their general spread and pattern did not differ among them. \\



\subsection{Comparison among Species}
The differences among species give interesting insights into the relationship between Kendall's $\tau$ and the species-specific role of resilience in forest decline. The fact that for some species resilience did not show a signficant effect but did for others suggests that other processes might take place that have not been captured in the given predictor set. While the trend in spatial variance showed a significant negative relationship with forest decline for \textit{Pinus halepensis}, spatial variance showed a positive relationship with forest decline for \textit{Quercus ilex}. As described above, this suggests that \textit{Pinus halepensis} becomes increasingly homogeneous before a decline event, while forests mostly consisting of \textit{Quercus ilex} become increasingly heterogeneous. This might be due to forest structure and spatial resolution on which it was calculated.\\
\cite{serra2018} found a negative effect of plot basal area on resilience for some drought years in the study area. Overall they attribute this to a competition-effect of taller trees that show different micro-environmental effects. But also the opposite effect is possible: taller trees and taller tree species within a plot might have a more extensive root system that could make them more resilient towards drought-stress. If several species occur within a given plot this might lead to a mixed signal. In any case, further research is needed to investigate this effect.\\
