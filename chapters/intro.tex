\section{Introduction}\label{sec:intro}
Despite the urgency of the topic, our knowledge on causes and mechanisms as well as our ability to reliably predict tree mortality and forest decline on individual to regional scale are still lacking \citep{mcdowell2013, hartmann2015}. Forest ecosystems worldwide have been observed to show accelerating rates of tree mortality linked to drought stress and increased temperatures \citep{allen2010, chaparro2017, vanmantgen2009}. As climate change advances, this process is likely to intensify \citep{allen2015}, putting entire ecosystems at risk to undergo regime shifts. Tree mortality affecting a large part of a forest stand is known as forest decline \citep{martinez2012}. Forests globally are increasingly at the risk of declining \citep{choat2012} or in some areas even to undergo a regime shift, such as tropical forests transforming into savannas \citep{hirota2011}. This risk grows tangible in changes of frequency, intensity, duration and timing of fires and droughts, but also by the introduction of species, outbreaks of pests and pathogens, hurricanes, windstorms, ice storms, or landslides \citep{dale2001}. The phenomenon of tree mortality may occur at the level of the organism, but the processes leading to forest decline are observable and describable across spatial, organizational, and temporal scales and therefore demand for interdisciplinary approaches \citep{hartmann2015}.\\
A large study that synthesized findings on radial growth patterns prior to tree mortality found that the mortality is preceded by reduced growth rates in about 84\% of the cases \citep{cailleret2017}, and thus adding a temporal component and changes in plant productivity to the list of potential indicators of tree mortality. Temporal changes in forest responses to perturbations can also be characterized by remotely sensed \gls{resilience} \citep{verbesselt2016}. Remotely sensed \gls{resilience} makes use of the concept of Early Warning Signals \citep[EWS,][]{carpenter2006, carpenter2008, dakos2008}. These are indicators of resilience that are extracted from time series of satellite observations. These have the potential to reflect decreasing stability \citep{kefi2013} or even the approaching of a tipping point in the ecosystem \citep{scheffer2009b} based on slowing down observed through space and time.\\
So called Early Warning Signals (EWS) could provide the potential to warn in time when an ecosystem approaches a bifurcation \citep{dakos2014}. On the other hand, EWS can also be seen as a 'warning signal for a potential decrease in stability', independently of whether changes are catastrophic or not \citep{kefi2013}. Although EWS might not outperform direct (field) measurements of resilience \citep{vbelzen2017}, they still hold a big potential for predictive modeling especially for larger scales, remote areas and particularly for the modeling of past events. For the above described problem to explain drought-induced forest decline, they support the predictability as it is expected that a less resilient forest is more vulnerable to drought. Remotely sensed resilience could then contribute to large-scale predictions, and to the monitoring of forest decline risk and finally offer the possibility to look back in time \citep{verbesselt2016}. Since neither full understanding, nor methods to warn in time but as well as no accurate predictions of forest decline occurrence are available at this point, this research has great potential for forest management, as based on this, forest managers could adapt their planning. This research focuses on the predictive modeling of forest decline using both indicators of forest resilience and environmental variables that have proven to indicate part of the occurrence of forest decline \citep{chaparro2017}.\\
