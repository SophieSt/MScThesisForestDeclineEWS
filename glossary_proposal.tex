\newglossaryentry{resilience}
{
    name=resilience,
    description={the rate of return to the equilibrium state after the ecosystem has been disturbed, can be expressed by the degree of temporal relation between observations \citep{telesca2006, telesca2008, zaccarelli2013, dakos2012}}
}

\newglossaryentry{resistence}
{
    name=resistence,
    description={quantification of the impact of a perturbation on the ecosystem property, hence the ability of the ecosystem to maintain its original state following an environmental perturbation; can be quantified based on the magnitude of the anomaly at the moment of perturbation \citep{lloret2007, vanruijven2010, vogel2012}}
}

\newglossaryentry{variance}
{
    name=variance,
    description={more general idea of ecosystem stability, standard deviation or coefficient of variation of the anomaly time series, large variance of ecosystem property when resistence is lower and when return to equilibirium state is slower \citep{pimm1984, tilman1994, telesca2006, lloret2007, vogel2012}}
}

\newglossaryentry{tippoint}
{
    name=tipping point,
    description={critical bifurcation point, describes the point at which a system undergoes a transition into an alternative stable state, i.e. the regime shifts \citep{scheffer2001}}
}

\newglossaryentry{flickering}
{
    name=flickering,
    description={increased probability that a system shifts temporarily between alternative basins of attraction \citep{dakos2012}}
}


\newglossaryentry{basinattraction}
{
    name=basin of attraction,
    description={initial conditions that lead to a particular state (equilibrium). given in slope, height and distance between the borders. slopes are steep when resilience and/or resistence is high \citep{scheffer2001}}
}